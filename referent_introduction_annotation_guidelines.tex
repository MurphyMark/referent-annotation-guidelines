% Options for packages loaded elsewhere
\PassOptionsToPackage{unicode}{hyperref}
\PassOptionsToPackage{hyphens}{url}
%
\documentclass[
]{book}
\usepackage{lmodern}
\usepackage{amssymb,amsmath}
\usepackage{ifxetex,ifluatex}
\ifnum 0\ifxetex 1\fi\ifluatex 1\fi=0 % if pdftex
  \usepackage[T1]{fontenc}
  \usepackage[utf8]{inputenc}
  \usepackage{textcomp} % provide euro and other symbols
\else % if luatex or xetex
  \usepackage{unicode-math}
  \defaultfontfeatures{Scale=MatchLowercase}
  \defaultfontfeatures[\rmfamily]{Ligatures=TeX,Scale=1}
\fi
% Use upquote if available, for straight quotes in verbatim environments
\IfFileExists{upquote.sty}{\usepackage{upquote}}{}
\IfFileExists{microtype.sty}{% use microtype if available
  \usepackage[]{microtype}
  \UseMicrotypeSet[protrusion]{basicmath} % disable protrusion for tt fonts
}{}
\makeatletter
\@ifundefined{KOMAClassName}{% if non-KOMA class
  \IfFileExists{parskip.sty}{%
    \usepackage{parskip}
  }{% else
    \setlength{\parindent}{0pt}
    \setlength{\parskip}{6pt plus 2pt minus 1pt}}
}{% if KOMA class
  \KOMAoptions{parskip=half}}
\makeatother
\usepackage{xcolor}
\IfFileExists{xurl.sty}{\usepackage{xurl}}{} % add URL line breaks if available
\IfFileExists{bookmark.sty}{\usepackage{bookmark}}{\usepackage{hyperref}}
\hypersetup{
  pdftitle={Referent Introduction Annotation Guidelines},
  hidelinks,
  pdfcreator={LaTeX via pandoc}}
\urlstyle{same} % disable monospaced font for URLs
\usepackage{longtable,booktabs}
% Correct order of tables after \paragraph or \subparagraph
\usepackage{etoolbox}
\makeatletter
\patchcmd\longtable{\par}{\if@noskipsec\mbox{}\fi\par}{}{}
\makeatother
% Allow footnotes in longtable head/foot
\IfFileExists{footnotehyper.sty}{\usepackage{footnotehyper}}{\usepackage{footnote}}
\makesavenoteenv{longtable}
\usepackage{graphicx,grffile}
\makeatletter
\def\maxwidth{\ifdim\Gin@nat@width>\linewidth\linewidth\else\Gin@nat@width\fi}
\def\maxheight{\ifdim\Gin@nat@height>\textheight\textheight\else\Gin@nat@height\fi}
\makeatother
% Scale images if necessary, so that they will not overflow the page
% margins by default, and it is still possible to overwrite the defaults
% using explicit options in \includegraphics[width, height, ...]{}
\setkeys{Gin}{width=\maxwidth,height=\maxheight,keepaspectratio}
% Set default figure placement to htbp
\makeatletter
\def\fps@figure{htbp}
\makeatother
\setlength{\emergencystretch}{3em} % prevent overfull lines
\providecommand{\tightlist}{%
  \setlength{\itemsep}{0pt}\setlength{\parskip}{0pt}}
\setcounter{secnumdepth}{5}
\usepackage{booktabs}
\usepackage{amsthm}
\makeatletter
\def\thm@space@setup{%
  \thm@preskip=8pt plus 2pt minus 4pt
  \thm@postskip=\thm@preskip
}
\makeatother
\usepackage[]{natbib}
\bibliographystyle{plainnat}

\title{Referent Introduction Annotation Guidelines}
\author{}
\date{\vspace{-2.5em}2020-06-05}

\begin{document}
\maketitle

{
\setcounter{tocdepth}{1}
\tableofcontents
}
\hypertarget{research-questions}{%
\chapter{Research Questions}\label{research-questions}}

\begin{enumerate}
\def\labelenumi{\arabic{enumi}.}
\item
  Do heratige speakers introduce more referents than monolinguals?
\item
  Do heratige speakers use different types of referents than monolinguals?
\item
  Do heratige speakers explain conjoined referents more often than monolinguals?
\end{enumerate}

\hypertarget{answer-1-referents}{%
\chapter{Answer 1: Referents}\label{answer-1-referents}}

\hypertarget{referents}{%
\section{Referents}\label{referents}}

(Annotated on tier \texttt{norm{[}referent{]}} in ExMARALDA)

We have a list of 20 possible referents, and we count how many of these 20 referents each speaker introduces:

\begin{enumerate}
\def\labelenumi{\arabic{enumi}.}
\tightlist
\item
  man (with the ball)
\item
  woman1 (with the stroller)
\item
  couple (man and woman1) - \emph{conjoined referent}
\item
  people (all the people in the parking lot) - \emph{conjoined referent}
\item
  family (man+woman1+baby) - \emph{conjoined referent}
\item
  ball
\item
  stroller
\item
  baby
\item
  woman2 (with the dog)
\item
  dog
\item
  leash
\item
  groceries
\item
  trunk
\item
  car1 (blue one, comes in first, gets hit)
\item
  car2 (white one, comes in second, hits car1)
\item
  car3 (red one, woman with groceries)
\item
  cars (car1 + car2) - \emph{conjoined referent}
\item
  driver1 (blue car, calls 911)
\item
  driver2 (white car)
\item
  drivers (driver1 + driver2) - \emph{conjoined referent}
\end{enumerate}

\hypertarget{man}{%
\subsection{man}\label{man}}

\hypertarget{woman1}{%
\subsection{woman1}\label{woman1}}

\hypertarget{couple}{%
\subsection{couple}\label{couple}}

\hypertarget{people}{%
\subsection{people}\label{people}}

\hypertarget{family}{%
\subsection{family}\label{family}}

\hypertarget{ball}{%
\subsection{ball}\label{ball}}

\hypertarget{stroller}{%
\subsection{stroller}\label{stroller}}

\hypertarget{baby}{%
\subsection{baby}\label{baby}}

\hypertarget{woman2}{%
\subsection{woman2}\label{woman2}}

\hypertarget{dog}{%
\subsection{dog}\label{dog}}

\hypertarget{leash}{%
\subsection{leash}\label{leash}}

\hypertarget{groceries}{%
\subsection{groceries}\label{groceries}}

\hypertarget{trunk}{%
\subsection{trunk}\label{trunk}}

\hypertarget{car1}{%
\subsection{car1}\label{car1}}

\hypertarget{car2}{%
\subsection{car2}\label{car2}}

\hypertarget{car3}{%
\subsection{car3}\label{car3}}

\hypertarget{cars}{%
\subsection{cars}\label{cars}}

\hypertarget{driver1}{%
\subsection{driver1}\label{driver1}}

\hypertarget{driver2}{%
\subsection{driver2}\label{driver2}}

\hypertarget{drivers}{%
\subsection{drivers}\label{drivers}}

\hypertarget{answer-2-r-type}{%
\chapter{Answer 2: R-Type}\label{answer-2-r-type}}

\hypertarget{r-type}{%
\section{R-Type}\label{r-type}}

\begin{quote}
Annotated on tier \texttt{norm{[}r-type{]}} in ExMARALDA
\end{quote}

We use the ReFlex annotation scheme to give each new referent a referential label.
The original ReFlex paper can be found at \url{https://elib.uni-stuttgart.de/handle/11682/9028}.

We use the following 7 labels:

\begin{longtable}[]{@{}lll@{}}
\toprule
& R-Type & Abbreviation\tabularnewline
\midrule
\endhead
1 & New & \texttt{n}\tabularnewline
2 & Unused-Unknown & \texttt{u}\tabularnewline
3 & Bridging & \texttt{b}\tabularnewline
4 & Bridging-Contained & \texttt{bc}\tabularnewline
5 & Bridging-Displaced & \texttt{bd}\tabularnewline
6 & Given & \texttt{g}\tabularnewline
7 & Given-Displaced & \texttt{gd}\tabularnewline
\bottomrule
\end{longtable}

\hypertarget{new}{%
\section{New}\label{new}}

A regular new referent, normally introduced with an indefinite article, referential ``this'', or a numeral.

\begin{quote}
I saw {[}this man{]} \texttt{new} walking down the street and {[}a woman{]} \texttt{new} with a stroller.
There were also {[}two cars{]} \texttt{new} coming.
\end{quote}

\begin{quote}
A woman \texttt{new} with a black dog \texttt{new} was loading groceries \texttt{new} into a minivan \texttt{new}.
\end{quote}

Questions:

\begin{itemize}
\tightlist
\item
  \protect\hyperlink{new-vs.-unused-unknown}{New vs.~Unused-Unknown}
\end{itemize}

\hypertarget{unused-unknown}{%
\section{Unused-Unknown}\label{unused-unknown}}

\begin{quote}
``Assigned to referring expressions which come with a sufficient amount of descriptive material to enable the hearer to create a new discourse referent without any previous knowledge'' (p.~4).
\end{quote}

A new referent followed by explanation.
The explanation must be a part of the noun phrase containing the referent, and must include a specific non-new referent.

\begin{quote}
I saw a man and a woman.
The ball \texttt{unused-unknown} that the man was dribbling fell on the ground.
\end{quote}

\begin{quote}
I was standing in the parking lot.
The two cars \texttt{unused-unkown} that were coming towards me stopped abruptly.
\end{quote}

Referents preceded by possessive pronouns are also unused-unkown.

\begin{quote}
I saw a man \texttt{new} with his wife \texttt{unused-unknown} and her baby \texttt{unused-unknown}.
\end{quote}

\hypertarget{bridging}{%
\section{Bridging}\label{bridging}}

\begin{quote}
``If an entity does not have a coreferential antecedent but can be understood as unique with respect to a previously introduced situation or scenario, we will be using the label r-bridging'' (p.~4).
\end{quote}

\begin{quote}
``This label is used for non-coreferential anaphoric expressions which are dependent on and unique with respect to a previously introduced scenario'' (p.~8).
\end{quote}

In other words, we use this tag for referents that have not been explicitly introduced but are implied as an essential part of an already known referent (the anchor).

If a referent was bridging, you would expect it to be there, and be surprised if it did not exist.
For something that could be there, but isn't necessarily (e.g.~a woman's groceries), use the unused-unknown.

\begin{quote}
There were two cars \texttt{new} coming, and they crashed into each other.
The drivers \texttt{bridging} got out and called the police.
\end{quote}

\begin{quote}
There was a family \texttt{new} walking down the street.
The father \texttt{bridging} was dribbling a ball.
\end{quote}

``Father'' is bridging because we know that if there is a family, there must be a father.
So, when ``father'' is introduced, you already implicitly know it's a member of that family.

\begin{quote}
I saw a car crash! The first car \texttt{bridging} stopped short, and the second \texttt{bridging} drove into it.
\end{quote}

Questions

\begin{itemize}
\tightlist
\item
  \protect\hyperlink{bridging-vs.-bridging-contained}{Bridging vs.~Bridging-Contained}
\end{itemize}

\hypertarget{bridging-contained}{%
\subsection{Bridging-Contained}\label{bridging-contained}}

\begin{quote}
``This label applies to a non-coreferential anaphoric expression that is anchored to an embedded phrase'' (p.~8).
\end{quote}

Briding-Contained is similar to Unused-Unknown, except that it requires the anchor of a bridging relationship in the same phrase, not just an introduced referent.

Examples:

\begin{quote}
The driver \texttt{bc} of the blue car \ldots{}
\end{quote}

\begin{quote}
The driver \texttt{bc} of the closer car \ldots{}
\end{quote}

\begin{quote}
The father \texttt{bc} of the family \ldots{}
\end{quote}

\begin{quote}
I saw a car crash! The driver \texttt{bc} who was in the car \texttt{b} got out.
\end{quote}

Since a car is expected to have a driver, the driver is bridging-contained.

\begin{quote}
I saw a car crash! The man \texttt{bc} who was in the car \texttt{b} got out.
\end{quote}

Questions:

\begin{itemize}
\tightlist
\item
  \protect\hyperlink{bridging-vs.-bridging-contained}{Bridging vs.~Bridging-Contained}
\end{itemize}

\hypertarget{bridging-displaced}{%
\subsection{Bridging-Displaced}\label{bridging-displaced}}

A Bridging-Displaced referent is a Bridging referent with 5 non-empty CU's between the CU containing the referent and the most recent CU containing its anchor.

\begin{quote}
There was a couple \ldots{}
{[}5 CU's{]} \ldots{}
The mother was just standing there and didn't do anything
\end{quote}

Questions:

\hypertarget{given}{%
\section{Given}\label{given}}

A given referent is one that has been previously introduced, either as itself, or as part of a given-relationship conjoined referent.

\begin{quote}
There was a man.
He \texttt{g} was dribbling a ball.
\end{quote}

\begin{quote}
There were these two people walking.
One \texttt{g} of them had a ball.
\end{quote}

\begin{quote}
There were two cars approaching the scene.
The first car \texttt{given} stopped, and the second car \texttt{given} bumped into it.
\end{quote}

\begin{quote}
There was a man who was dribbling a ball and a woman with a stroller.
The couple \texttt{given} was crossing the street.
\end{quote}

Questions:

\hypertarget{given-displaced}{%
\subsection{Given-Displaced}\label{given-displaced}}

A Given-Displaced referent is a Given referent with 5 non-empty CU's between the CU containing the referent and the most recent CU containing the same referent.

\begin{quote}
There was a man \ldots{}
{[} 5 CU's {]} \ldots{}
The man \texttt{gd} was running to catch the ball.
\end{quote}

\begin{quote}
A man and a woman were walking.
There were two cars coming \ldots{}
{[} 5 CU's {]} \ldots{}
The man \texttt{gd} helped the lady with her groceries.
\end{quote}

Questions:

\hypertarget{answer-3-conjoined-referents}{%
\chapter{Answer 3: Conjoined Referents}\label{answer-3-conjoined-referents}}

\hypertarget{questions-answers}{%
\chapter{Questions \& Answers}\label{questions-answers}}

\hypertarget{referents-1}{%
\section{Referents}\label{referents-1}}

\hypertarget{possessive-pronoun-whose}{%
\subsection{Possessive Pronoun ``whose''}\label{possessive-pronoun-whose}}

\begin{quote}
How do we tag ``whose'' in referent noun phrases such as ``There was a woman whose car \ldots{}''
\end{quote}

Tag ``whose'' as part of the referent that comes after.

\begin{quote}
There was a woman {[}whose car{]} \texttt{u} was very spacious.
\end{quote}

\hypertarget{identifying-repititions}{%
\subsection{Identifying Repititions}\label{identifying-repititions}}

\begin{quote}
Does this example contain a repitition?
\end{quote}

Here are some previous questions about repitions, and our decisions.
The referents in question are marked with \texttt{{[}{]}} brackets:

\begin{quote}
A woman was closing {[}her trunk{]}, opening {[}it{]}, and \ldots{}
\end{quote}

Tag both referents normally.

\begin{quote}
The car behind him - hit {[}it{]} also - hit {[}the car{]} - {[}the other car{]}
\end{quote}

Tag only ``the other car''.

\begin{quote}
And the guy with {[}the ball{]} who accidentally dropped {[}the ball{]} \ldots{}
\end{quote}

Tag only the second ``the ball''.

\hypertarget{tagging-unclear-referents}{%
\subsection{Tagging Unclear Referents}\label{tagging-unclear-referents}}

\begin{quote}
How confusing can a description be before we decide not to tag the referent?
\end{quote}

Here are some examples, and our decisions.
The referents in question are marked with \texttt{{[}{]}} brackets:

\begin{quote}
I saw a lady, and {[}a boy{]} next to her bouncing a ball.
\end{quote}

\texttt{man} - \texttt{u}

\begin{quote}
\ldots{} so the {[}owner{]} of {[}the cars{]} called 911.
\end{quote}

Neither is tagged, because there is no single owner of two cars.

\begin{quote}
The two people came out of their cars and {[}a man{]} with the white shirt called the police.
\end{quote}

\texttt{driver} - \texttt{g}

\begin{quote}
{[}They{]} both came out, and {[}one of them{]} called the police
\end{quote}

``They'': \texttt{drivers} - \texttt{g}

``One of them'': \texttt{driver1} - \texttt{g}

\begin{quote}
I think {[}someone{]} like called 911
\end{quote}

\texttt{drivers}

\begin{quote}
The man helped the woman. Uh, {[}she{]} helped her pick up her groceries.
\end{quote}

\texttt{man}

\begin{quote}
\ldots{} which distracted a dog that was in the road behind {[}a car{]}.
\end{quote}

Untagged

\begin{quote}
{[}The passengers{]} got out and called the police.
\end{quote}

Untagged

\begin{quote}
The car behind him was coming after him, so {[}he{]} crashed into {[}that car{]} too.
\end{quote}

Both untagged

\hypertarget{r-types}{%
\section{R-Types}\label{r-types}}

\hypertarget{new-vs.-unused-unknown}{%
\subsection{New vs.~Unused-Unknown}\label{new-vs.-unused-unknown}}

\begin{quote}
Should I tag this referent as New or Unused-Unknown?
\end{quote}

Tag as \texttt{u} if:

\begin{itemize}
\tightlist
\item
  The predicate contains a \protect\hyperlink{phrases-containing-given-referents}{given referent}
\item
  The predicate contains a bridging or unused-unknown referent
\end{itemize}

Otherwise, tag as \texttt{n}.

\hypertarget{examples}{%
\subsubsection{Examples}\label{examples}}

\hypertarget{phrases-containing-given-referents}{%
\subsection{Phrases Containing Given Referents}\label{phrases-containing-given-referents}}

\begin{quote}
Does this phrase contain a given referent?
\end{quote}

A phrase contains a given referent if:

\begin{itemize}
\item
  There is an annotated referent marked \texttt{g}
\item
  There is a non-annotated referent that has been previously introduced
\end{itemize}

Examples:

\begin{quote}
The man who was holding the ball \texttt{g} and talking to a woman dropped it.
\end{quote}

Phrase: ``who was holding the ball''

``the ball'': Given

``a woman'': New

Phrase contains a given referent

\begin{quote}
The man who was holding a ball \texttt{n} dropped it.
\end{quote}

Phrase: ``who was holding a ball''

``a ball'': New

Phrase does not contain a given referent

\begin{quote}
The man with the woman \texttt{g} 's ball \texttt{u} dropped it.
\end{quote}

Phrase: ``with the woman's ball''

``the woman'': Given

``'s ball'': Unused-Unknown

Phrase contains a given referent

\begin{quote}
A couple was walking down the road.
A woman across the street was unloading groceries from her car.
\end{quote}

Phrase: ``across the street''

``the street'': Given

Phrase contains a given referent

Note: Although ``the street'' is not annotated, it has been introduced in the narrative.
So, it is considered a given referent.

\begin{quote}
A couple was walking with a stroller.
A woman across the street was unloading groceries from her car.
\end{quote}

Phrase: ``across the street''

``the street'': New

Phrase does not contain a given referent

\hypertarget{bridging-vs.-bridging-contained}{%
\subsection{Bridging vs.~Bridging-Contained}\label{bridging-vs.-bridging-contained}}

\begin{quote}
Should I tag this referent as Bridging or Bridging-Contained?
\end{quote}

Simalar to \protect\hyperlink{new-vs.-unused-unknown}{unused-unknown}, tag as \texttt{bc} if:

\begin{itemize}
\item
  The anchor of the referent appears in the predicate
\item
  The referent is a part of a possessive phrase with the anchor
\end{itemize}

\hypertarget{examples-1}{%
\subsubsection{Examples}\label{examples-1}}

\hypertarget{possessive-pronoun-multiple-referents}{%
\subsection{Possessive Pronoun Multiple Referents}\label{possessive-pronoun-multiple-referents}}

\begin{quote}
Can a possessive pronoun give unused-unknown status to multiple coordinated referents?
\end{quote}

Yes, if the possessive pronoun applies to a coordinated noun phrase, all referents in that noun phrase are unused-unknown.

\begin{quote}
There was a woman with {[}her dog{]} \texttt{u} and \protect\hyperlink{groceries}{groceries} \texttt{u}.
\end{quote}

\hypertarget{counting-cus}{%
\subsection{Counting CU's}\label{counting-cus}}

\begin{quote}
How do we count the 5 CU's between a referent and its most recent co-referent/anchor
\end{quote}

An r-type is \texttt{-Displaced} if there are 5 CU's \emph{between} the CU containing the referent and the CU containing its most recent co-referent/anchor.

The count of CU's should exclude \href{empty-cu's}{empty CU's}.

\hypertarget{empty-cus}{%
\subsection{Empty CU's}\label{empty-cus}}

\begin{quote}
Which CU's are empty CU's?
\end{quote}

Empty CU's are those without words - pauses, tongue clicks, gulping, etc.

\hypertarget{introductions}{%
\subsection{Introductions}\label{introductions}}

\hypertarget{phrases}{%
\section{Phrases}\label{phrases}}

\hypertarget{does-one-behind-the-other-have-referents}{%
\subsection{Does ``one behind the other'' Have Referents?}\label{does-one-behind-the-other-have-referents}}

\end{document}
