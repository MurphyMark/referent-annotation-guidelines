% Options for packages loaded elsewhere
\PassOptionsToPackage{unicode}{hyperref}
\PassOptionsToPackage{hyphens}{url}
%
\documentclass[
]{book}
\usepackage{lmodern}
\usepackage{amssymb,amsmath}
\usepackage{ifxetex,ifluatex}
\ifnum 0\ifxetex 1\fi\ifluatex 1\fi=0 % if pdftex
  \usepackage[T1]{fontenc}
  \usepackage[utf8]{inputenc}
  \usepackage{textcomp} % provide euro and other symbols
\else % if luatex or xetex
  \usepackage{unicode-math}
  \defaultfontfeatures{Scale=MatchLowercase}
  \defaultfontfeatures[\rmfamily]{Ligatures=TeX,Scale=1}
\fi
% Use upquote if available, for straight quotes in verbatim environments
\IfFileExists{upquote.sty}{\usepackage{upquote}}{}
\IfFileExists{microtype.sty}{% use microtype if available
  \usepackage[]{microtype}
  \UseMicrotypeSet[protrusion]{basicmath} % disable protrusion for tt fonts
}{}
\makeatletter
\@ifundefined{KOMAClassName}{% if non-KOMA class
  \IfFileExists{parskip.sty}{%
    \usepackage{parskip}
  }{% else
    \setlength{\parindent}{0pt}
    \setlength{\parskip}{6pt plus 2pt minus 1pt}}
}{% if KOMA class
  \KOMAoptions{parskip=half}}
\makeatother
\usepackage{xcolor}
\IfFileExists{xurl.sty}{\usepackage{xurl}}{} % add URL line breaks if available
\IfFileExists{bookmark.sty}{\usepackage{bookmark}}{\usepackage{hyperref}}
\hypersetup{
  pdftitle={Referent Introduction Annotation Guidelines},
  hidelinks,
  pdfcreator={LaTeX via pandoc}}
\urlstyle{same} % disable monospaced font for URLs
\usepackage{longtable,booktabs}
% Correct order of tables after \paragraph or \subparagraph
\usepackage{etoolbox}
\makeatletter
\patchcmd\longtable{\par}{\if@noskipsec\mbox{}\fi\par}{}{}
\makeatother
% Allow footnotes in longtable head/foot
\IfFileExists{footnotehyper.sty}{\usepackage{footnotehyper}}{\usepackage{footnote}}
\makesavenoteenv{longtable}
\usepackage{graphicx,grffile}
\makeatletter
\def\maxwidth{\ifdim\Gin@nat@width>\linewidth\linewidth\else\Gin@nat@width\fi}
\def\maxheight{\ifdim\Gin@nat@height>\textheight\textheight\else\Gin@nat@height\fi}
\makeatother
% Scale images if necessary, so that they will not overflow the page
% margins by default, and it is still possible to overwrite the defaults
% using explicit options in \includegraphics[width, height, ...]{}
\setkeys{Gin}{width=\maxwidth,height=\maxheight,keepaspectratio}
% Set default figure placement to htbp
\makeatletter
\def\fps@figure{htbp}
\makeatother
\setlength{\emergencystretch}{3em} % prevent overfull lines
\providecommand{\tightlist}{%
  \setlength{\itemsep}{0pt}\setlength{\parskip}{0pt}}
\setcounter{secnumdepth}{5}
\usepackage{booktabs}
\usepackage{amsthm}
\makeatletter
\def\thm@space@setup{%
  \thm@preskip=8pt plus 2pt minus 4pt
  \thm@postskip=\thm@preskip
}
\makeatother
\usepackage[]{natbib}
\bibliographystyle{plainnat}

\title{Referent Introduction Annotation Guidelines}
\author{}
\date{\vspace{-2.5em}2020-06-05}

\begin{document}
\maketitle

{
\setcounter{tocdepth}{1}
\tableofcontents
}
\hypertarget{research-questions}{%
\chapter{Research Questions}\label{research-questions}}

\begin{enumerate}
\def\labelenumi{\arabic{enumi}.}
\item
  Do heratige speakers introduce more referents than monolinguals?
\item
  Do heratige speakers use different types of referents than monolinguals?
\item
  Do heratige speakers explain conjoined referents more often than monolinguals?
\end{enumerate}

\hypertarget{answer-1-referents}{%
\chapter{Answer 1: Referents}\label{answer-1-referents}}

\hypertarget{referents}{%
\section{Referents}\label{referents}}

(Annotated on tier \texttt{norm{[}referent{]}} in ExMARALDA)

We have a list of 20 possible referents, and we count how many of these 20 referents each speaker introduces:

\begin{enumerate}
\def\labelenumi{\arabic{enumi}.}
\tightlist
\item
  man (with the ball)
\item
  woman1 (with the stroller)
\item
  couple (man and woman1) - \emph{conjoined referent}
\item
  people (all the people in the parking lot) - \emph{conjoined referent}
\item
  family (man+woman1+baby) - \emph{conjoined referent}
\item
  ball
\item
  stroller
\item
  baby
\item
  woman2 (with the dog)
\item
  dog
\item
  leash
\item
  groceries
\item
  trunk
\item
  car1 (blue one, comes in first, gets hit)
\item
  car2 (white one, comes in second, hits car1)
\item
  car3 (red one, woman with groceries)
\item
  cars (car1 + car2) - \emph{conjoined referent}
\item
  driver1 (blue car, calls 911)
\item
  driver2 (white car)
\item
  drivers (driver1 + driver2) - \emph{conjoined referent}
\end{enumerate}

See also:
* \protect\hyperlink{questionsux2freferents}{Referent Questions}

\hypertarget{answer-2-r-type}{%
\chapter{Answer 2: R-Type}\label{answer-2-r-type}}

\hypertarget{r-type}{%
\section{R-Type}\label{r-type}}

\begin{quote}
Annotated on tier \texttt{norm{[}r-type{]}} in ExMARALDA
\end{quote}

We use the ReFlex annotation scheme to give each new referent a referential label.
The original ReFlex paper can be found at \url{https://elib.uni-stuttgart.de/handle/11682/9028}.

We use the following 7 labels:

\begin{longtable}[]{@{}lll@{}}
\toprule
& R-Type & Abbreviation\tabularnewline
\midrule
\endhead
1 & New & \texttt{n}\tabularnewline
2 & Unused-Unknown & \texttt{u}\tabularnewline
3 & Bridging & \texttt{b}\tabularnewline
4 & Bridging-Contained & \texttt{bc}\tabularnewline
5 & Bridging-Displaced & \texttt{bd}\tabularnewline
6 & Given & \texttt{g}\tabularnewline
7 & Given-Displaced & \texttt{gd}\tabularnewline
\bottomrule
\end{longtable}

\hypertarget{new}{%
\section{New}\label{new}}

A regular new referent, normally introduced with an indefinite article, referential ``this'', or a numeral.

\begin{quote}
I saw {[}this man{]} \texttt{new} walking down the street and {[}a woman{]} \texttt{new} with a stroller.
There were also {[}two cars{]} \texttt{new} coming.
\end{quote}

\begin{quote}
A woman \texttt{new} with a black dog \texttt{new} was loading groceries \texttt{new} into a minivan \texttt{new}.
\end{quote}

Questions:

\begin{itemize}
\tightlist
\item
  \protect\hyperlink{new-vs.-unused-unknown}{New vs.~Unused-Unknown}
\end{itemize}

\hypertarget{unused-unknown}{%
\section{Unused-Unknown}\label{unused-unknown}}

\begin{quote}
``Assigned to referring expressions which come with a sufficient amount of descriptive material to enable the hearer to create a new discourse referent without any previous knowledge'' (p.~4).
\end{quote}

A new referent followed by explanation.
The explanation must be a part of the noun phrase containing the referent, and must include a specific non-new referent.

\begin{quote}
I saw a man and a woman.
The ball \texttt{unused-unknown} that the man was dribbling fell on the ground.
\end{quote}

\begin{quote}
I was standing in the parking lot.
The two cars \texttt{unused-unkown} that were coming towards me stopped abruptly.
\end{quote}

Referents preceded by possessive pronouns are also unused-unkown.

\begin{quote}
I saw a man \texttt{new} with his wife \texttt{unused-unknown} and her baby \texttt{unused-unknown}.
\end{quote}

\hypertarget{bridging}{%
\section{Bridging}\label{bridging}}

\begin{quote}
``If an entity does not have a coreferential antecedent but can be understood as unique with respect to a previously introduced situation or scenario, we will be using the label r-bridging'' (p.~4).
\end{quote}

\begin{quote}
``This label is used for non-coreferential anaphoric expressions which are dependent on and unique with respect to a previously introduced scenario'' (p.~8).
\end{quote}

In other words, we use this tag for referents that have not been explicitly introduced but are implied as an essential part of an already known referent (the anchor).

If a referent was bridging, you would expect it to be there, and be surprised if it did not exist.
For something that could be there, but isn't necessarily (e.g.~a woman's groceries), use the unused-unknown.

\begin{quote}
There were two cars \texttt{new} coming, and they crashed into each other.
The drivers \texttt{bridging} got out and called the police.
\end{quote}

\begin{quote}
There was a family \texttt{new} walking down the street.
The father \texttt{bridging} was dribbling a ball.
\end{quote}

``Father'' is bridging because we know that if there is a family, there must be a father.
So, when ``father'' is introduced, you already implicitly know it's a member of that family.

\begin{quote}
I saw a car crash! The first car \texttt{bridging} stopped short, and the second \texttt{bridging} drove into it.
\end{quote}

Questions

\begin{itemize}
\tightlist
\item
  \protect\hyperlink{bridging-vs.-bridging-contained}{Bridging vs.~Bridging-Contained}
\end{itemize}

\hypertarget{bridging-contained}{%
\subsection{Bridging-Contained}\label{bridging-contained}}

\begin{quote}
``This label applies to a non-coreferential anaphoric expression that is anchored to an embedded phrase'' (p.~8).
\end{quote}

Briding-Contained is similar to Unused-Unknown, except that it requires the anchor of a bridging relationship in the same phrase, not just an introduced referent.

Examples:

\begin{quote}
The driver \texttt{bc} of the blue car \ldots{}
\end{quote}

\begin{quote}
The driver \texttt{bc} of the closer car \ldots{}
\end{quote}

\begin{quote}
The father \texttt{bc} of the family \ldots{}
\end{quote}

\begin{quote}
I saw a car crash! The driver \texttt{bc} who was in the car \texttt{b} got out.
\end{quote}

Since a car is expected to have a driver, the driver is bridging-contained.

\begin{quote}
I saw a car crash! The man \texttt{bc} who was in the car \texttt{b} got out.
\end{quote}

Questions:

\begin{itemize}
\tightlist
\item
  \protect\hyperlink{bridging-vs.-bridging-contained}{Bridging vs.~Bridging-Contained}
\end{itemize}

\hypertarget{bridging-displaced}{%
\subsection{Bridging-Displaced}\label{bridging-displaced}}

A Bridging-Displaced referent is a Bridging referent with 5 non-empty CU's between the CU containing the referent and the most recent CU containing its anchor.

\begin{quote}
There was a couple \ldots{}
{[}5 CU's{]} \ldots{}
The mother was just standing there and didn't do anything
\end{quote}

Questions:

\hypertarget{given}{%
\section{Given}\label{given}}

A given referent is one that has been previously introduced, either as itself, or as part of a given-relationship conjoined referent.

\begin{quote}
There was a man.
He \texttt{g} was dribbling a ball.
\end{quote}

\begin{quote}
There were these two people walking.
One \texttt{g} of them had a ball.
\end{quote}

\begin{quote}
There were two cars approaching the scene.
The first car \texttt{given} stopped, and the second car \texttt{given} bumped into it.
\end{quote}

\begin{quote}
There was a man who was dribbling a ball and a woman with a stroller.
The couple \texttt{given} was crossing the street.
\end{quote}

Questions:

\hypertarget{given-displaced}{%
\subsection{Given-Displaced}\label{given-displaced}}

A Given-Displaced referent is a Given referent with 5 non-empty CU's between the CU containing the referent and the most recent CU containing the same referent.

\begin{quote}
There was a man \ldots{}
{[} 5 CU's {]} \ldots{}
The man \texttt{gd} was running to catch the ball.
\end{quote}

\begin{quote}
A man and a woman were walking.
There were two cars coming \ldots{}
{[} 5 CU's {]} \ldots{}
The man \texttt{gd} helped the lady with her groceries.
\end{quote}

Questions:

\hypertarget{answer-3-conjoined-referents}{%
\chapter{Answer 3: Conjoined Referents}\label{answer-3-conjoined-referents}}

\hypertarget{questions-answers}{%
\chapter{Questions \& Answers}\label{questions-answers}}

\hypertarget{referents-1}{%
\section{Referents}\label{referents-1}}

\hypertarget{groceries}{%
\subsection{Groceries}\label{groceries}}

\begin{quote}
What phrases should we tag as \texttt{groceries}? What phrases should we not?
\end{quote}

Generally, \texttt{groceries} should be a mass noun.
Here's a list of things we've decided to tag as \texttt{groceries}, and a list of things we've decided not to.

Groceries:

\begin{itemize}
\tightlist
\item
  groceries
\item
  produce
\item
  shopping
\item
  stuff
\item
  food
\end{itemize}

Not Groceries:

\begin{itemize}
\tightlist
\item
  bags
\item
  baggage
\item
  fruits
\end{itemize}

\hypertarget{family-vs.-couple}{%
\subsection{Family vs.~Couple}\label{family-vs.-couple}}

\begin{quote}
Is this group of people \texttt{family} or \texttt{couple} ?
\end{quote}

Generally, if all the members of the family have been introduced, they refers to the family.
If only man and woman1 (and maybe stroller) have been introduced, they refers to the couple, instead.

Also: if there's a man and a woman \emph{with} a baby, we treat it as a couple, as the baby is just additional information about either the man or the woman.

\begin{quote}
There was a man, a woman, and a baby.
They \texttt{family} crossed the street.
\end{quote}

\begin{quote}
There was a man and a woman.
They \texttt{couple} crossed the street.
\end{quote}

\begin{quote}
There was a man and a woman and a stroller.
They \texttt{couple} went across the street.
\end{quote}

\begin{quote}
There was a man and a woman with a baby.
They \texttt{couple} crossed the street.
\end{quote}

\hypertarget{possessive-pronoun-whose}{%
\subsection{Possessive Pronoun ``whose''}\label{possessive-pronoun-whose}}

\begin{quote}
How do we tag ``whose'' in referent noun phrases such as ``There was a woman whose car \ldots{}''
\end{quote}

Tag ``whose'' as part of the referent that comes after.

\begin{quote}
There was a woman {[}whose car{]} \texttt{u} was very spacious.
\end{quote}

\hypertarget{repititions}{%
\subsection{Repititions}\label{repititions}}

\begin{quote}
What do we do with repeated references (``A ball - a ball was rolling \ldots{}'')?
\end{quote}

If a referent was repeated, tag only the last referent - the one the speaker eventually decided on.

\begin{quote}
A man \texttt{untagged} was, a man \texttt{n} was walking down the street.
\end{quote}

\hypertarget{identifying-repititions}{%
\subsection{Identifying Repititions}\label{identifying-repititions}}

\begin{quote}
Does this example contain a repitition?
\end{quote}

Here are some previous questions about repitions, and our decisions.
The referents in question are marked with \texttt{{[}{]}} brackets:

\begin{quote}
A woman was closing {[}her trunk{]}, opening {[}it{]}, and \ldots{}
\end{quote}

Tag both referents normally.

\begin{quote}
The car behind him - hit {[}it{]} also - hit {[}the car{]} - {[}the other car{]}
\end{quote}

Tag only ``the other car''.

\begin{quote}
And the guy with {[}the ball{]} who accidentally dropped {[}the ball{]} \ldots{}
\end{quote}

Tag only the second ``the ball''.

\hypertarget{tagging-unclear-referents}{%
\subsection{Tagging Unclear Referents}\label{tagging-unclear-referents}}

\begin{quote}
How confusing can a description be before we decide not to tag the referent?
\end{quote}

Here are some examples, and our decisions.
The referents in question are marked with \texttt{{[}{]}} brackets:

\begin{quote}
I saw a lady, and {[}a boy{]} next to her bouncing a ball.
\end{quote}

\texttt{man} - \texttt{u}

\begin{quote}
\ldots{} so the {[}owner{]} of {[}the cars{]} called 911.
\end{quote}

Neither is tagged, because there is no single owner of two cars.

\begin{quote}
The two people came out of their cars and {[}a man{]} with the white shirt called the police.
\end{quote}

\texttt{driver} - \texttt{g}

\begin{quote}
{[}They{]} both came out, and {[}one of them{]} called the police
\end{quote}

``They'': \texttt{drivers} - \texttt{g}

``One of them'': \texttt{driver1} - \texttt{g}

\begin{quote}
I think {[}someone{]} like called 911
\end{quote}

\texttt{drivers}

\begin{quote}
The man helped the woman. Uh, {[}she{]} helped her pick up her groceries.
\end{quote}

\texttt{man}

\begin{quote}
\ldots{} which distracted a dog that was in the road behind {[}a car{]}.
\end{quote}

Untagged

\begin{quote}
{[}The passengers{]} got out and called the police.
\end{quote}

Untagged

\begin{quote}
The car behind him was coming after him, so {[}he{]} crashed into {[}that car{]} too.
\end{quote}

Both untagged

\hypertarget{generic-referents}{%
\subsection{Generic Referents}\label{generic-referents}}

\begin{quote}
Is this referent generic, and should we tag it?
\end{quote}

Here are some examples, and our decisions:

\begin{quote}
The woman was holding her dog by {[}the leash{]}.
\end{quote}

\texttt{leash} - \texttt{b}

\begin{quote}
Both drivers got out of {[}the car{]}
\end{quote}

Untagged

\hypertarget{aposotrophe-s}{%
\subsection{Aposotrophe S}\label{aposotrophe-s}}

\begin{quote}
What do we do with 's phrases, like ``the man's ball''?
\end{quote}

Normally when you see an apostrophe s, you can break the phrase up into a given and an unused-unknown.

However, if there is a bridging relationship between the two members of the phrase, the second member might be bridging-contained instead.
This requires a bridging relationship, though, and most apostrophe s's will be unused-unknown.

\begin{quote}
A woman was getting her groceries.
The woman \texttt{g} {[}`s dog{]} \texttt{u} ran into the road.
\end{quote}

\begin{quote}
A car pulled over.
The car \texttt{g} {[}`s driver{]} \texttt{bc} got out, and started yelling.
\end{quote}

\hypertarget{possessive-pronouns}{%
\subsection{Possessive Pronouns}\label{possessive-pronouns}}

\begin{quote}
What do we do with phrases containing possessive pronouns (his ball)?
\end{quote}

Possessive pronoun phrases are either unused-unknown or bridging (never bridging-contained).
If there is a bridging relationship, then bridging. Otherwise unused-unknown.

\begin{quote}
A woman was in the parking lot.
{[}Her groceries{]} \texttt{u} fell.
\end{quote}

\begin{quote}
One of the drivers stopped suddenly.
{[} Their car {]} \texttt{b} screeched really loud.
\end{quote}

\hypertarget{wh-words-and-possessive-pronouns}{%
\subsection{Wh-Words and Possessive Pronouns}\label{wh-words-and-possessive-pronouns}}

\begin{quote}
Should we tag wh-words or possessive pronouns that are not attached to a referent?
\end{quote}

Nope! Only personal pronouns and referent phrases should be tagged.

\begin{quote}
A man \texttt{n} was in the parking lot.
He \texttt{g} was happy.
\end{quote}

\begin{quote}
A man \texttt{n} was in the parking lot.
In his \texttt{untagged} hand was {[}his ball{]} \texttt{u}.
\end{quote}

\begin{quote}
There was a woman.
The man \texttt{u} who \texttt{untagged} was with her looked confused.
\end{quote}

\hypertarget{conjoined-referents}{%
\subsection{Conjoined Referents}\label{conjoined-referents}}

\begin{quote}
When a conjoined referent appears, are members of it treated as given or bridging?
Also, when members of a conjoined referent appear, is the conjoined referent treated as given or bridging?
\end{quote}

If the conjoined referent is just the collection of individual referents, members are treated as given.
If all members have been previously listed, the conjoined referent is given, too.

Given relationship conjoined referents:

\begin{itemize}
\tightlist
\item
  cars
\item
  drivers
\end{itemize}

If the conjoined referent has some inferred bridging relationship between it and its members, members are treated as bridging.
If all members have been previously listed, the conjoined referent is bridging.

However, if the conjoined referent is introduced in a way that enumerates the members (``Two people were walking down the street''),
introduced members are treated as Given.

Bridging relationship conjoined referents:

\begin{itemize}
\tightlist
\item
  couple
\item
  family
\end{itemize}

\begin{quote}
A red car and a blue car drove down the street.
The cars \texttt{g} were going way too fast.
\end{quote}

\begin{quote}
Two cars were driving down the street.
The red one \texttt{g} was going slow.
\end{quote}

\begin{quote}
One car's driver got out.
Then the other car's driver got out.
The drivers \texttt{g} started arguing.
\end{quote}

\begin{quote}
The drivers got out of the car.
The first driver \texttt{g} looked very angry.
\end{quote}

\begin{quote}
Two people were walking down the street.
One of them \texttt{g} was bouncing a ball.
\end{quote}

\begin{quote}
A man and a woman were standing around.
The couple \texttt{b} was holding hands.
\end{quote}

\begin{quote}
A couple was standing in the parking lot.
The husband \texttt{b} was looking around.
\end{quote}

\begin{quote}
A man, a woman, and a baby were standing in the parking lot.
The family \texttt{b} looked happy.
\end{quote}

\begin{quote}
A family was walking down the street.
The baby \texttt{b} was crying.
\end{quote}

\hypertarget{r-types}{%
\section{R-Types}\label{r-types}}

\hypertarget{new-vs.-unused-unknown}{%
\subsection{New vs.~Unused-Unknown}\label{new-vs.-unused-unknown}}

\begin{quote}
Should I tag this referent as New or Unused-Unknown?
\end{quote}

Tag as \texttt{u} if:

\begin{itemize}
\tightlist
\item
  The predicate contains a \protect\hyperlink{phrases-containing-given-referents}{given referent}
\item
  The predicate contains a bridging or unused-unknown referent
\end{itemize}

Otherwise, tag as \texttt{n}.

\hypertarget{phrases-containing-given-referents}{%
\subsection{Phrases Containing Given Referents}\label{phrases-containing-given-referents}}

\begin{quote}
Does this phrase contain a given referent?
\end{quote}

A phrase contains a given referent if:

\begin{itemize}
\item
  There is an annotated referent marked \texttt{g}
\item
  There is a non-annotated referent that has been previously introduced
\end{itemize}

Examples:

\begin{quote}
The man who was holding the ball \texttt{g} and talking to a woman dropped it.
\end{quote}

Phrase: ``who was holding the ball''

``the ball'': Given

``a woman'': New

Phrase contains a given referent

\begin{quote}
The man who was holding a ball \texttt{n} dropped it.
\end{quote}

Phrase: ``who was holding a ball''

``a ball'': New

Phrase does not contain a given referent

\begin{quote}
The man with the woman \texttt{g} 's ball \texttt{u} dropped it.
\end{quote}

Phrase: ``with the woman's ball''

``the woman'': Given

``'s ball'': Unused-Unknown

Phrase contains a given referent

\begin{quote}
A couple was walking down the road.
A woman across the street was unloading groceries from her car.
\end{quote}

Phrase: ``across the street''

``the street'': Given

Phrase contains a given referent

Note: Although ``the street'' is not annotated, it has been introduced in the narrative.
So, it is considered a given referent.

\begin{quote}
A couple was walking with a stroller.
A woman across the street was unloading groceries from her car.
\end{quote}

Phrase: ``across the street''

``the street'': New

Phrase does not contain a given referent

\hypertarget{bridging-vs.-bridging-contained}{%
\subsection{Bridging vs.~Bridging-Contained}\label{bridging-vs.-bridging-contained}}

\begin{quote}
Should I tag this referent as Bridging or Bridging-Contained?
\end{quote}

Simalar to \protect\hyperlink{new-vs.-unused-unknown}{unused-unknown}, tag as \texttt{bc} if either:

\begin{itemize}
\tightlist
\item
  The anchor of the referent appears in the same phrase as the referent
\item
  The referent is a part of a possessive phrase with the anchor
\end{itemize}

\begin{quote}
A child was walking down the street.
The father \texttt{bc} of the child was following close behind.
\end{quote}

\begin{quote}
Two cars stopped in the street.
The first car {[}'s driver{]} \texttt{bc} got out angrily.
\end{quote}

\begin{quote}
The first car stopped in the street.
The driver \texttt{b} got out angrily.
\end{quote}

Since this example doesn't have ``the first car'' in the same phrase as ``the driver'', it is not Bridging-Contained.

\hypertarget{possessive-pronoun-multiple-referents}{%
\subsection{Possessive Pronoun Multiple Referents}\label{possessive-pronoun-multiple-referents}}

\begin{quote}
Can a possessive pronoun give unused-unknown status to multiple coordinated referents?
\end{quote}

Yes, if the possessive pronoun applies to a coordinated noun phrase, all referents in that noun phrase are unused-unknown.

\begin{quote}
There was a woman with {[}her dog{]} \texttt{u} and \protect\hyperlink{groceries}{groceries} \texttt{u}.
\end{quote}

\hypertarget{counting-cus}{%
\subsection{Counting CU's}\label{counting-cus}}

\begin{quote}
How do we count the 5 CU's between a referent and its most recent co-referent/anchor
\end{quote}

An r-type is \texttt{-Displaced} if there are 5 CU's \emph{between} the CU containing the referent and the CU containing its most recent co-referent/anchor.

The count of CU's should exclude \href{empty-cu's}{empty CU's}.

\hypertarget{empty-cus}{%
\subsection{Empty CU's}\label{empty-cus}}

\begin{quote}
Which CU's are empty CU's?
\end{quote}

Empty CU's are those without words - pauses, tongue clicks, gulping, etc.

\hypertarget{introductions}{%
\subsection{Introductions}\label{introductions}}

\hypertarget{conjoined-referents-1}{%
\section{Conjoined Referents}\label{conjoined-referents-1}}

\hypertarget{explaining-conjoined-referent-pronouns}{%
\subsection{Explaining Conjoined Referent Pronouns}\label{explaining-conjoined-referent-pronouns}}

\begin{quote}
If a pronoun refers to a conjoined referent (``The two cars stopped. {[}They{]} didn't stop fast enough''),
should we mark whether or not it's explained?
\end{quote}

Yep.
It's possible, for someone to explain a conjoined referent, even if it's a pronoun, so we'll just mark all conjoined referents.

\begin{quote}
There was a family walking down the street.
They \texttt{family} \texttt{g} \texttt{1}, a man, a woman, and a child, were walking quickly.
\end{quote}

\hypertarget{phrases}{%
\section{Phrases}\label{phrases}}

\hypertarget{one-behind-the-other}{%
\subsection{One Behind The Other}\label{one-behind-the-other}}

\begin{quote}
Does thre phrase ``one behind the other'' have any referents?
\end{quote}

No.
We are treating ``one behind the other'' as an idiom, so it has no referents.

\hypertarget{everyone}{%
\subsection{Everyone}\label{everyone}}

\begin{quote}
Which referent does ``everyone'' refer to?
\end{quote}

Here are some previous questions about ``everyone'', and our decisions:

\begin{quote}
The car hit into the back of another car.
\protect\hyperlink{everyone}{Everyone} seemed fine.
\end{quote}

Untagged

\begin{quote}
The man checked with \protect\hyperlink{everyone}{everyone} involved in the accident.
\end{quote}

\texttt{drivers}

\begin{quote}
{[}Everybody{]} got out and {[}they{]} were talking.
\end{quote}

\texttt{drivers} for both

\begin{quote}
\ldots{} then {[}everybody{]} got out and was yelling.
\end{quote}

\texttt{drivers}

\hypertarget{eitherneithernone-of-the}{%
\subsection{Either/Neither/None of the}\label{eitherneithernone-of-the}}

\begin{quote}
What do we do with ``Either / Neither / None of the {[} conjoined referent {]}'' ?
\end{quote}

Since either, neither, and none don't really refer to a concrete referent, we will not tag the phrase,
and we will not tag any referents within the phrase.

\begin{quote}
None of the family \texttt{untagged} looked very happy.
\end{quote}

\begin{quote}
Neither of the drivers \texttt{untagged} was at fault.
\end{quote}

\begin{quote}
Either one of them \texttt{untagged} could have ended the fight.
\end{quote}

\hypertarget{both-of-the}{%
\subsection{Both of the}\label{both-of-the}}

What do we do with ``Both of the {[} conjoined referent {]}'' ?

This phrase refers to the conjoined referent. Tag the whole phrase, and tag it with whatever the conjoined referent would be on its own.

\begin{quote}
The two cars stopped.
{[}Both of the drivers{]} \texttt{drivers} \texttt{b} got out.
\end{quote}

\begin{quote}
{[}Both of the people{]} \texttt{couple} \texttt{u} who had been walking down the street looked shocked.
\end{quote}

\hypertarget{one-of-the}{%
\subsection{One of the}\label{one-of-the}}

\begin{quote}
What do we do with ``One of the {[}conjoined referent{]}'' ?
\end{quote}

This phrase refers to a single member of the conjoined referent.
If the context gives evidence that it is one or the other of the members, tag the whole phrase as that member.

If there is no evidence to the identity of the referent, treat it like other ambiguous referents, and do not tag it.

\begin{quote}
The white car \texttt{g} crashed into the blue car \texttt{g}.
One of the drivers \texttt{driver} \texttt{b} got out and called the police.
\end{quote}

We know that the referent is driver1, because driver1 is the one who calls the police in this scenario.
We know that the referent is bridging, because driver1 has a bridging relationship with the blue car.

\begin{quote}
The two drivers \texttt{g} crashed into each other.
One of them \texttt{untagged} looked upset.
\end{quote}

Both drivers could be described as looking upset.
Since we don't know which driver the phrase refers to, we don't tag anything in the phrase.

\hypertarget{each-other}{%
\subsection{Each Other}\label{each-other}}

\begin{quote}
Should we tag ``each other'' ?
\end{quote}

Nope.

\hypertarget{technical}{%
\section{Technical}\label{technical}}

\hypertarget{hide-tiers}{%
\subsection{Hide Tiers}\label{hide-tiers}}

\begin{quote}
EXMARaLDA has too many tiers. Can I hide some?
\end{quote}

Yes! To hide tiers in Exmaralda, without deleting them:

\begin{itemize}
\tightlist
\item
  Select a tier on the left, or use shift-click to select multiple tiers
\item
  In the top menu, go to Tier \textgreater{} Hide Tier
\end{itemize}

\hypertarget{merge-cells}{%
\subsection{Merge Cells}\label{merge-cells}}

\begin{quote}
Is there a way to merge cells and immediately begin typing?
\end{quote}

Yes.
If you use the merge-cells button, you have to click on the merged cell to type into it.
However, if you instead use the merge-cells shortcut (ctrl-1 on Windows), you can immediately begin typing in the merged cell.

\end{document}
